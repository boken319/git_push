\documentclass[a4paper, 12pt]{article}
% A4纸张,article类型
\usepackage[UTF8]{ctex}
%以此支持中文
\usepackage{graphicx}
%以此插入图片
\usepackage{url}
%支持使用 \url命令
\usepackage{float}
\usepackage{xcolor}
\usepackage{hyperref} % 加载hyperref宏包 


\begin{document}
  \title{第一节课实验报告}
  \author{王书一 \\ 23020007119}
  \date{August 23, 2024}
  \maketitle

 \pagenumbering{arabic}
 \tableofcontents
 \newpage
 \pagenumbering{arabic}
 %目录

  
  \section{实验目的}
   通过实践掌握 Git 的基本操作,理解其在版本控制中的重要性,学会如何使用 Git 进行代码管理。
  \section{实验环境配置}
  \subsection{Git 安装与配置}
  1. 在 https://git-scm.com 下载并安装 Git。
  
  2. 配置用户名和邮箱:
  \begin{verbatim}
  git config --global user.name "你的名字"
  git config --global user.email "你的邮箱"
  \end{verbatim}
  
  \begin{figure}[h]
  \centering
    \includegraphics[width=0.7\linewidth]{配置用户名和邮箱.png}
  \caption{配置用户名和邮箱}
   \end{figure}

  \subsection{连接远程仓库}
  配置 SSH 密钥,连接远程仓库(例如 GitHub 或 GitLab)。执行以下命令生成 SSH 密钥:
  \begin{verbatim}
  ssh-keygen -t rsa -C "自己的邮箱" 
  \end{verbatim}

\begin{figure}[H]
  \centering
    \includegraphics[width=0.7\linewidth]{ssh.png}
  \caption{SSH 密钥生成示意图}
   \label{fig:ssh-keygen} 
   \end{figure}

\begin{figure}[H]
  \centering
 \includegraphics[width=0.7\linewidth]{配置好的远程仓库.png}
  \caption{配置好的远程仓库}
   \end{figure}

配置完成后,执行以下命令测试ssh连接
\begin{verbatim}
 ssh -T git@github.com
  \end{verbatim}
  
\begin{figure}[H]
  \centering
 \includegraphics[width=0.7\linewidth]{测试SSH连接.png}
  \caption{测试ssh连接}
   \end{figure}
   
  \newpage
  \section{git的使用}
  \subsection{初始化仓库}
  1. 创建项目文件夹并进入:
  \begin{verbatim}
  mkdir GitExperiment
  cd GitExperiment
  \end{verbatim}
  
  2. 初始化 Git 仓库:
  \begin{verbatim}
  git init
  \end{verbatim}
  
  \begin{figure}[h]
  \centering
 \includegraphics[width=0.7\linewidth]{创建本地仓库.png}
  \caption{创建本地仓库}
   \end{figure}
   
  \subsection{克隆远程至本地}
  \begin{verbatim}
  git clone git@github.com:用户名/仓库名.git
  \end{verbatim}
  
    \begin{figure}[h]
  \centering
 \includegraphics[width=0.7\linewidth]{克隆远程仓库.png}
  \caption{克隆远程仓库}
   \end{figure}
   
     \begin{figure}[H]
  \centering
 \includegraphics[width=0.7\linewidth]{克隆成功.png}
  \caption{克隆成功}
   \end{figure}
  
  \subsection{文件的添加与提交}
  1. 创建新文件:
  \begin{verbatim}
  echo "测试用例" > experiment.txt
  \end{verbatim}

  2. 添加文件到暂存区:
  \begin{verbatim}
  git add experiment.txt
  \end{verbatim}

    \begin{figure}[H]
  \centering
 \includegraphics[width=0.7\linewidth]{添加文件到暂存区.png}
  \caption{添加文件到暂存区}
   \end{figure}

  3.添加远程仓库并查看
   \begin{verbatim}
 git remote add origin https://github.com/username/repository.git
 git remote -v
   \end{verbatim}

    \begin{figure}[H]
  \centering
 \includegraphics[width=0.7\linewidth]{添加远程仓库.png}
  \caption{添加远程仓库}
   \end{figure}
   
  4. 提交文件:
  \begin{verbatim}
  git commit -m "第一次提交:添加实验报告文件"
  git push -u origin master
  \end{verbatim}

      \begin{figure}[H]
  \centering
 \includegraphics[width=0.7\linewidth]{提交成功.png}
  \caption{提交成功}
   \end{figure}

       \begin{figure}[H]
  \centering
 \includegraphics[width=0.7\linewidth]{操作后的结果.png}
  \caption{操作后的结果}
   \end{figure}

  \subsection{分支的创建与合并}
  1. 创建并切换到新分支:
  \begin{verbatim}
  git branch new-branch 创建新分支,名为new-branch
  git checkout new-branch 将HEAD切换到new-branch
  \end{verbatim}

  2. 在新分支上进行修改并提交:
  \begin{verbatim}
  echo this is a new-branch > new test.txt 新建txt文件
  echo "在分支上进行修改" >> experiment.txt
  git add experiment.txt
  git add new test.txt
  git commit -m"on new branch to test" 提交
  \end{verbatim}

    \begin{figure}[H]
  \centering
 \includegraphics[width=0.7\linewidth]{创建新分支并进行修改提交.png}
  \caption{创建新分支并进行修改提交}
   \end{figure}

  3. 切换回主分支并合并:
  \begin{verbatim}
  git checkout main
  git merge feature-branch
  \end{verbatim}

   \begin{figure}[H]
  \centering
 \includegraphics[width=0.7\linewidth]{切换回主分支并合并.png}
  \caption{切换回主分支并合并}
   \end{figure}

  \begin{figure}[H]
  \centering
 \includegraphics[width=0.7\linewidth]{修改后的结果.png}
  \caption{修改后的结果}
   \end{figure}

   \begin{figure}[H]
  \centering
 \includegraphics[width=0.7\linewidth]{上传到github的结果.png}
  \caption{上传到github的结果}
   \end{figure}
   
  \subsection{撤销对工作区文件的修改}
  切换回本地仓库Gitexperiment,对文件进行修改后执行撤销操作,再检查其中的内容。
   \begin{verbatim}
 git checkout -- 文件名 撤销工作区文件的修改
git checkout -- experiment.txt 
  \end{verbatim}
  
 \begin{figure}[H]
  \centering
 \includegraphics[width=0.7\linewidth]{修改及撤回的操作.png}
  \caption{修改及撤回的操作}
   \end{figure}

 \subsection{撤销对缓存区的修改}
  将新修改的工作区添加进入缓存区后,进行撤销操作。
   \begin{verbatim}
git reset HEAD 加入缓存区的文件名 撤销缓存区的修改
git status 查看当前git的状态
  \end{verbatim}
  
 \begin{figure}[H]
  \centering
 \includegraphics[width=0.7\linewidth]{撤销对缓存区的修改.png}
  \caption{撤销对缓存区的修改}
   \end{figure}

\section{Latex的使用}
\subsection{文件的基本格式和标题的生成}
 \begin{verbatim}
\documentclass{article}
\title{实验报告}
\author{作者姓名}
\date{\today}
\begin{document}
\maketitle
  \end{verbatim}
  
\subsection{目录的生成}
   \begin{verbatim}
\tableofcontents
\newpage
  \end{verbatim}
  
\subsection{章节的生成}
\begin{verbatim}
\section{章节的标题}
这里是章节的内容。
\subsection{子章节标题}
这里是子章节的内容。
  \end{verbatim}
  
\subsection{彩色字体的使用}
 \begin{verbatim}
\usepackage{xcolor}
\textcolor{blue}{这是蓝色字体}
  \end{verbatim}
  

\textcolor{blue}{使用蓝色字体的实例}
\colorbox{yellow}{\color{red}黄底红字的示例}

\subsection{自定义页码}
\begin{verbatim}
\pagenumbering{Roman} % 使用罗马数字页码
  \end{verbatim}
  
\subsection{列表(清单)的生成}
\begin{verbatim}
\begin{itemize}
    \item 第一项
    \item 第二项
\end{itemize}

\begin{enumerate}
    \item 第一步
    \item 第二步
\end{enumerate}
  \end{verbatim}
第一种列表
\begin{itemize}
    \item 第一项
    \item 第二项
\end{itemize}

第二种列表
   \begin{enumerate}
  \item First thing
  \item Second thing
    \begin{itemize}
      \item A sub-thing
      \item Another sub-thing
    \end{itemize}
\end{enumerate}

  
\subsection{表格的生成}
\begin{verbatim}
\begin{tabular}{|c|c|c|}
\hline
列1 & 列2 & 列3 \\
\hline
数据1 & 数据2 & 数据3 \\
\hline
\end{tabular}
  \end{verbatim}

左对齐|右对齐|右对齐
  \begin{tabular}{l|r|r}
  Item   & Quantity  & Price(\$)\\
  \hline
  Nails  & 500       & 0.34\\
  Wooden boards &100 & 4.00\\
  Bricks & 240       & 11.50\\
\end{tabular}

  
\subsection{图表的使用}
\begin{verbatim}
\usepackage{graphicx}
\begin{figure}[h]
    \centering
    \includegraphics[width=0.5\textwidth]{example.png}
    \caption{图的标题}
\end{figure}
  \end{verbatim}


\begin{figure}[H]
    \centering
    \includegraphics[width=0.5\textwidth]{示例图片.png}
    \caption{示例图片}
\end{figure}
  
\subsection{插入各种数学公式}
\begin{verbatim}
\begin{equation}
    E = mc^2
\end{equation}
  \end{verbatim}

插入公式
  \begin{equation}
    E = mc^2
\end{equation}
  
\subsection{书写参考文献}
\begin{verbatim}
\begin{thebibliography}{99}
    \bibitem{ref1} 作者, 书名, 出版社, 出版年份.
\end{thebibliography}
  \end{verbatim}



  \newpage
  \section{git实验总结}
  通过本次实验,掌握了 Git 的基本使用方法,包括初始化仓库、文件的提交、分支的创建与合并,以及版本回退。这些功能对于项目的版本管理非常重要。

  \section{Latex实验总结}

在本次实验中,我深入学习并实践了 LaTeX 的基本使用方法。通过实验,我掌握了 LaTeX 文档的基本结构,包括如何生成标题、章节、目录以及各种格式控制。具体而言,实验内容涵盖了以下几个方面:

\begin{itemize}
    \item 学习了文件的基本格式及标题、章节的生成;
    \item 熟悉了 LaTeX 中目录的生成方式;
    \item 通过练习掌握了如何使用不同颜色的字体以及自定义页码;
    \item 了解了 LaTeX 中列表、表格、图表的生成方法;
    \item 学习了插入数学公式以及参考文献书写的规范。
\end{itemize}

通过本次实验,我不仅熟悉了 LaTeX 的基础功能,还体会到了其在处理复杂文档结构中的强大优势。在自动生成目录、格式规范化、数学公式编写以及图表插入等方面,LaTeX 显示出了其在学术写作中的不可替代性。同时,在实验中也遇到了一些挑战,例如排版控制和包的兼容性问题。

总的来说,本次实验有效提升了我使用 LaTeX 进行学术报告和文档撰写的能力,为后续的学术研究和文档编写奠定了坚实的基础。


  \section{参考文献}
  \begin{enumerate}
      \item Git 官方文档: \url{https://git-scm.com/doc}
      \item Pro Git 电子书: \url{https://git-scm.com/book/zh/v2}
      \item Latex文档:\url{https://www.overleaf.com/learn/latex/Learn_LaTeX_in_30_minutes}
  \end{enumerate}
  
 \section{代码链接}
      请访问以下链接来查看我在Github上的相关练习、报告和代码:
\url{https://github.com/boken319/git_push}


\end{document}
