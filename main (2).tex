\documentclass[a4paper, 12pt]{article}
% A4纸张,article类型
\usepackage[UTF8]{ctex}
%以此支持中文
\usepackage{graphicx}
%以此插入图片
\usepackage{url}
%支持使用 \url命令
\usepackage{float}
\usepackage{xcolor}
\usepackage{hyperref} % 加载hyperref宏包 
\usepackage{listings}  % 代码高亮包
\usepackage{amsmath}
\usepackage{verbatim}

\begin{document}
  \title{第三节课实验报告}
  \author{王书一 \\ 23020007119}
  \date{September 9, 2024}
  \maketitle

 \pagenumbering{arabic}
 \tableofcontents
 \newpage
 \pagenumbering{arabic}
 %目录

  
  \section{实验目的}
  掌握命令行环境工具的使用,特别是运行、终止等工具。
学习并应用Python进行基础数据处理和可视化。
  
\section{实验环境}
\begin{itemize}
    \item 操作系统:Ubuntu 20.04
   \item 编程语言:Python 3.12.0
\end{itemize}

\section{命令行环境}

\subsection{暂停和后台执行进程}


\begin{itemize}
    \item \texttt{Ctrl-Z}       让 shell 发送 SIGTSTP 信号,SIGTSTP 是 Terminal Stop 的缩写(即 terminal 版本的 SIGSTOP)
    \item \texttt{fg}        恢复暂停的工作,在前台继续
    \item \texttt{bg}        恢复暂停的工作,在后台继续
    \item \texttt{jobs}     列出当前终端会话中尚未完成的全部任务
\end{itemize}

\begin{figure}[H]
  \centering
    \includegraphics[width=0.9\linewidth]{3.png}
  \caption{暂停和后台执行进程}
   \end{figure}

\subsection{查找PID并执行命令}  

\begin{itemize}
  \item \texttt{pgrep}    查找当前运行的进程的命令行工具。它通过进程的名称或其他属性来搜索进程,并返回这些进程的进程ID(PID)。-l 选项告诉 pgrep 在输出中不仅包含进程的 PID,还要包含进程的完整命令行。
   \item \texttt{\&}    命令中的 \& 后缀可以让命令在直接在后台运行
  \item \texttt{kill [选项] 进程ID}    kill命令用于终止指定的进程(terminate a process)的运行
  \item \texttt{kill -STOP 进程ID}    暂停进程
 \item \texttt{kill -CONT 进程ID}    恢复进程
  \end{itemize}

SIGTERM (15):

默认信号,请求进程终止。进程可以捕获此信号并进行清理操作。
命令:kill PID 或 kill -15 PID

SIGKILL (9):

强制终止进程,进程无法捕获或忽略此信号。
命令:kill -9 PID
  
\begin{figure}[H]
  \centering
    \includegraphics[width=0.9\linewidth]{3.png}
  \caption{查找PID并执行命令}
   \end{figure}

\subsection{别名}  

大多数 shell 都支持设置别名。shell 的别名相当于一个长命令的缩写,shell 会自动将其替换成原本的命令。

\begin{lstlisting}
ls -lAh --time=atime --color=auto
# 创建常用命令的缩写
alias ll="ls -lh"
alias gs="git status"
alias v="vim"
# 手误打错命令也没关系
alias sl=ls
# 重新定义一些命令行的默认行为
alias mv="mv -i"           # -i prompts before overwrite
alias mkdir="mkdir -p"     # -p make parent dirs as needed
alias df="df -h"           # -h prints human readable format
# 获取别名的定义
alias ll
# 会打印 ll='ls -lh'
\end{lstlisting}

\begin{figure}[H]
  \centering
    \includegraphics[width=0.9\linewidth]{2.png}
  \caption{别名的应用}
   \end{figure}

问题1:创建一个 dc 别名,它的功能是当我们错误的将 cd 输入为 dc 时也能正确执行。
\begin{lstlisting}
alias cd="dc"
\end{lstlisting}
注意:等号两边不可以加空格

如果需要保存此别名需要对文件进行更改
\begin{lstlisting}
步骤 1:编辑 .bashrc 或 .bash_profile 文件  nano ~/.bashrc
步骤 2:添加别名   在 .bashrc 文件中添加以下行来创建 dc 别名:
alias dc='cd'
步骤 3:重新加载 .bashrc 文件  source ~/.bashrc
步骤4:验证别名  可以尝试输入 dc 来验证别名是否生效:
dc /path/to/directory
\end{lstlisting}

\section{Python入门基础}
\subsection{计算器代码}  
写一个简单的命令行计算器,用户可以输入两个数字并选择一个操作(加、减、乘、除),然后程序会输出计算结果。


\begin{lstlisting}
编写思路
用户交互:通过 input 函数获取用户输入的两个数字和操作符。
条件判断:使用 if-elif-else 结构来判断用户选择的操作符,
并执行相应的计算。
错误处理:在除法操作中,检查除数是否为零,避免程序崩溃。
输出结果:使用 print 函数输出计算结果。
\end{lstlisting}

\begin{figure}[H]
  \centering
    \includegraphics[width=0.9\linewidth]{4.png}
  \caption{计算器代码}
   \end{figure}

   \begin{figure}[H]
  \centering
    \includegraphics[width=0.9\linewidth]{5.png}
  \caption{计算器运行结果}
   \end{figure}
升级方案:
循环执行:允许用户多次进行计算,而不是只执行一次。
当用户输入“q”时退出
\begin{figure}[H]
  \centering
    \includegraphics[width=0.9\linewidth]{6.png}
  \caption{计算器代码}
   \end{figure}
   
  \subsection{猜数字小游戏}  
  写一个猜数字游戏,使用Python编写。游戏的目的是让玩家猜测一个由程序随机生成的位于1到100(包括1和100)之间的整数。
\begin{lstlisting}
导入random模块:首先,通过import random语句导入了Python的random模块,
这个模块提供了生成随机数的功能。
定义guess_number函数:接着,定义一个名为guess_number的函数,
这个函数不接受任何参数,并且会执行猜数字游戏的逻辑。
生成随机数:在函数内部,首先使用random.randint(1, 100)生成一个1到100
之间的随机整数,并将其赋值给变量number。这个随机数就是玩家需要猜测的数字。
初始化尝试次数:然后,初始化一个名为attempts的变量,并将其设置为0。
这个变量用于记录玩家猜测的次数。
无限循环:使用while True创建一个无限循环,这个循环会一直执行,
直到玩家猜对数字为止。
获取玩家猜测:在循环内部,首先通过input函数获取玩家的输入,
并使用int函数将其转换为整数,赋值给变量guess。
判断猜测结果:
调用函数:最后,通过guess_number()调用这个函数,启动游戏。
\end{lstlisting}

\begin{figure}[H]
  \centering
    \includegraphics[width=0.9\linewidth]{7.png}
  \caption{猜数字代码}
   \end{figure}
   
\begin{figure}[H]
  \centering
    \includegraphics[width=0.9\linewidth]{8.png}
  \caption{运行结果}
   \end{figure}   
   
\subsection{文件处理}  
这段代码定义了一个名为的函数,它接受一个文件路径作为参数,并计算并打印出该文件中每个单词出现的频率

读取并处理文本:
file.read() 读取文件的全部内容。
.lower() 将文本转换为小写,以确保单词的频率统计不区分大小写(例如,"Hello" 和 "hello" 会被视为同一个单词)。
.split() 将文本分割成一个单词列表。默认情况下,split() 方法按空白字符(如空格、换行符等)分割字符串。
计算单词频率:使用 Counter(text) 创建一个 Counter 对象,该对象会统计列表中每个元素(这里是单词)的出现次数。
打印单词频率:遍历 Counter 对象的项(单词及其频率),并使用 print(f"{word}: {freq}") 打印每个单词及其对应的频率


\begin{figure}[H]
  \centering
    \includegraphics[width=0.9\linewidth]{9.png}
  \caption{代码}
   \end{figure}   

   \begin{figure}[H]
  \centering
    \includegraphics[width=0.9\linewidth]{10.png}
  \caption{运行结果}
   \end{figure}   

\subsection{写一个备忘录}  
定义一个简单的文本备忘录程序,它允许用户添加、查看、删除备忘,并提供了退出程序的功能。程序使用一个列表 notes 来存储所有的备忘。

代码解释:
初始化备忘列表:程序开始时,创建一个空列表 notes 来存储备忘。
主循环:程序进入一个无限循环,直到用户选择退出(选项4)。
显示操作选项:向用户显示四个操作选项:添加备忘、查看备忘、删除备忘、退出。
用户输入处理:
添加备忘(选项1):接收用户输入的备忘内容,并将其添加到 notes 列表中。
查看备忘(选项2):遍历 notes 列表,并打印出所有备忘及其编号(编号从1开始,但列表索引从0开始,因此打印时需要将索引加1)。
删除备忘(选项3):接收用户输入的备忘编号,将其转换为整数并减1以匹配列表索引,然后检查该索引是否有效。如果有效,则从 notes 列表中删除对应的备忘。
退出(选项4):退出程序。
错误处理:如果用户输入了无效的操作选项,程序会提示用户重新输入。

 
\begin{figure}[H]
  \centering
    \includegraphics[width=0.9\linewidth]{11.png}
  \caption{代码}
   \end{figure}  

    \begin{figure}[H]
  \centering
    \includegraphics[width=0.9\linewidth]{12.png}
  \caption{运行结果}
   \end{figure}  
    \begin{figure}[H]
  \centering
    \includegraphics[width=0.9\linewidth]{13.png}
  \caption{运行结果}
   \end{figure}   
   
\subsection{实现一个剪刀石头布游戏}  

总体思路
定义游戏选项:首先,定义一个包含所有可能游戏选项(石头、剪刀、布)的列表。
游戏循环:使用一个无限循环(while True)来允许用户连续玩游戏,直到他们选择退出。
计算机选择:在每次循环中,使用random.choice()函数从游戏选项列表中随机选择一个作为计算机的选择。
用户输入:提示用户输入他们的选择,并允许他们输入“退出”来结束游戏。
输入验证:检查用户的输入是否有效(即是否是游戏选项之一或“退出”)。如果输入无效,则提示用户重新输入,并继续当前循环。
比较结果:如果输入有效,则比较用户的选择和计算机的选择,以确定胜负或平局。
输出结果:根据比较结果,打印出相应的消息(如“你赢了!”、“你输了!”或“平局!”)。
退出条件:如果用户输入“退出”,则打印结束消息并退出循环,从而结束游戏。

详细步骤
初始化:
创建一个包含“石头”、“剪刀”和“布”的列表choices。
导入random模块以使用random.choice()函数。
游戏循环:
使用while True创建一个无限循环。
计算机选择:
在每次循环开始时,使用random.choice(choices)随机选择一个选项作为计算机的选择。
用户输入:
提示用户输入他们的选择,并允许他们输入“退出”来结束游戏。
使用input()函数获取用户的输入。
输入验证:
检查用户的输入是否在choices列表中或是否等于“退出”(不区分大小写)。
如果输入无效,打印错误消息并使用continue跳过当前循环的剩余部分,回到循环的开始。
比较结果:
如果输入有效,则使用一系列的if-elif-else语句来比较用户的选择和计算机的选择。
根据比较结果,打印出相应的胜负或平局消息。
输出结果:
使用print()函数打印出比较结果的消息。
退出条件:
如果用户输入“退出”,则打印结束消息并使用break退出循环,从而结束游戏。

\begin{figure}[H]
  \centering
    \includegraphics[width=0.9\linewidth]{14.png}
  \caption{代码}
   \end{figure}  

    \begin{figure}[H]
  \centering
    \includegraphics[width=0.9\linewidth]{15.png}
  \caption{运行结果}
   \end{figure}  

解决方案:
该代码仍与常规游戏平局继续循环,判断输赢后结束不符,故进行修改
\begin{lstlisting}
   # 比较用户和计算机的选择  
        if user_choice == computer_choice:  
            print(f"平局!都选择了{user_choice},请继续...")  
        elif (user_choice == '石头' and computer_choice == '剪刀') or \  
             (user_choice == '剪刀' and computer_choice == '布') or \  
             (user_choice == '布' and computer_choice == '石头'):  
            print(f"你赢了!{user_choice}打败了{computer_choice},
            游戏结束。")  
            break  # 如果用户赢了,则退出循环结束游戏  
        else:  
            print(f"你输了!{computer_choice}打败了{user_choice},
            游戏结束。")  
            break  # 如果用户输了,也退出循环结束游戏 
\end{lstlisting} 

\subsection{生成随机密码} 
编写一段代码,定义一个名为 generate\_password 的函数,该函数用于生成指定长度的随机密码。使用了 Python 的 random 和 string 模块来实现其功能。

函数参数
length: 一个整数,指定了生成的密码应该包含多少个字符。
函数内部逻辑
定义字符集:首先,通过组合 string.ascii\_letters(包含所有ASCII字母,大小写各26个),string.digits(包含所有10个数字),和 string.punctuation(包含所有ASCII标点符号),构建了一个包含所有可用字符的字符串 characters。这样,生成的密码可以包含字母、数字和标点符号,增强了密码的复杂性。
生成密码:使用列表推导式和 random.choice() 函数,从 characters 字符串中随机选择 length 个字符,并将它们连接成一个字符串。这个字符串就是生成的随机密码。
返回密码:函数最后返回这个生成的密码字符串。
编写思路
明确需求:首先明确需要生成一个随机密码,密码的长度应该是可配置的。
选择字符集:为了增加密码的复杂度,选择了包含字母、数字和标点符号的字符集。
生成随机序列:利用 Python 的 random 模块中的 choice() 函数,结合列表推导式,从字符集中随机选择指定数量的字符。
返回结果:将生成的随机字符序列连接成一个字符串,并返回。
运行结果
当调用 generate\_password(12) 时,函数会生成一个长度为12的随机密码,并打印出来。 
\begin{lstlisting}
import random
import string

def generate_password(length):
    characters = string.ascii_letters + string.digits + string.punctuation
    password = ''.join(random.choice(characters) for i in range(length))
    return password

# 调用示例
# print("生成的密码:", generate_password(12))
运行结果:生成的密码:
b3J$2#M8n@4%
\end{lstlisting}

\subsection{进行数组排序}
代码需求:
生成一串随机数组,使用快速排序进行排序
代码思路:
实现quicksort 函数,需要遵循快速排序算法的基本思想:选择一个基准值(pivot),然后将数组分为小于基准值的左部分、等于基准值的中间部分和大于基准值的右部分,之后递归地对左右两部分进行快速排序,最后将这三部分合并

\begin{figure}[H]
  \centering
    \includegraphics[width=0.9\linewidth]{16.png}
  \caption{代码及运行结果}
   \end{figure}  

 代码缺陷及修改:
   使用arr.copy() 来生成原始数组 arr 的一个副本,并将其传递给 quicksort 函数。这样做是为了防止在排序过程中修改原始数组 arr。

示例使用  
\begin{lstlisting}
    import random  
    arr = [random.randint(0, 100) for _ in range(20)]  
    print("原始数组:", arr)  
    sorted_arr = quicksort(arr.copy()) 
    # 使用 arr.copy() 来避免修改原始数组  
    print("排序后的数组:", sorted_arr)  
注意:此时 arr 仍然是原始数组,sorted_arr 是排序后的新数组

\end{lstlisting}
    


\section{Python视觉应用}


\subsection{环境配置}
PIL(Python Imaging Library Python,图像处理类库)提供了通用的图像处理功能,以及大量有用的基本图像操作,比如图像缩放、裁剪、旋转、颜色转换等。

通过网络查询发现,现在更常用PIL的更现代、更活跃的分支Pillow。Pillow是PIL的一个友好分支,提供了强大的图像处理能力,并修复了PIL中的一些bug,增加了许多新特性。

我使用的是Python 3,需要使用pip3而不是pip来确保为Python 3安装Pillow。

pip3 install Pillow
from PIL import Image  这样调用PIL库

\subsection{打开并转换图像格式}
通过 save() 方法,PIL 可以将图像保存成多种格式的文件。下面的例子将图像文件,并转换成 JPEG 格式

\begin{lstlisting}
   # 打开图像
    image = Image.open('example.png')
    # 将图像模式从RGBA转换为RGB
    image = image.convert('RGB')
    # 'RGB' 是目标模式,表示图像将转换为红、绿、蓝三个通道的图像。
    # RGBA 模式表示图像有红、绿、蓝和透明度四个通道。
   # 尝试遍历一个名为 filelist 的列表,该列表包含要处理的图像文件的路径。对于每个文件,将其转换为 .jpg 格式
\end{lstlisting}

\begin{figure}[H]
  \centering
    \includegraphics[width=0.9\linewidth]{17.png}
  \caption{代码}
   \end{figure}  

\subsection{获取图片信息}
     获取图像信息
    width, height = image.size
    mode = image.mode
    print(f"Width: {width}, Height: {height},  Mode: {mode}")

\begin{figure}[H]
  \centering
    \includegraphics[width=0.5\linewidth]{18.png}
  \caption{运行结果}
   \end{figure}  



\subsection{创建缩略图}
使用 PIL 可以很方便地创建图像的缩略图。thumbnail() 方法接受一个元组参数(该
参数指定生成缩略图的大小),然后将图像转换成符合元组参数指定大小的缩略图。
例如,创建最长边为 128 像素的缩略图,可以使用下列命令:
pil\_im.thumbnail((128,128))


\begin{verbatim}
     # 生成缩略图
    thumbnail_size = (100, 100)
    image.thumbnail(thumbnail_size)
    image.save('thumbnail_example.jpg')
    print("缩略图已生成并保存为 thumbnail_example.jpg")
  \end{verbatim}

\begin{figure}[H]
  \centering
    \includegraphics[width=0.9\linewidth]{example.png}
  \caption{example.png}
   \end{figure}  

   \begin{figure}[H]
  \centering
    \includegraphics[width=0.9\linewidth]{thumbnail_example.jpg}
  \caption{缩略图}
   \end{figure}  





\subsection{裁剪、调整尺寸和旋转}
要调整一幅图像的尺寸,我们可以调用 resize() 方法。该方法的参数是一个元组,
用来指定新图像的大小
要旋转一幅图像,可以使用逆时针方式表示旋转角度,然后调用 rotate() 方法

\begin{verbatim}
  # 裁剪图像
    box = (100, 100, 400, 400)
    cropped_image = image.crop(box)
    cropped_image.show()

    # 调整图像大小
    new_size = (300, 300)
    resized_image = image.resize(new_size)
    resized_image.show()

    # 旋转图像
    rotated_image = image.rotate(45)
    rotated_image.show()
  \end{verbatim}

\begin{figure}[H]
  \centering
    \includegraphics[width=0.9\linewidth]{19.png}
  \caption{裁剪后的图片}
   \end{figure}  
   
   \begin{figure}[H]
  \centering
    \includegraphics[width=0.9\linewidth]{20.png}
  \caption{改变尺寸的图片}
   \end{figure}  
   
   \begin{figure}[H]
  \centering
    \includegraphics[width=0.9\linewidth]{21.png}
  \caption{旋转后的图片}
   \end{figure}  
   

\subsection{绘制图像、点和线}
尽管 Matplotlib 可以绘制出较好的条形图、饼状图、散点图等,但是对于大多数计
算机视觉应用来说,仅仅需要用到几个绘图命令。最重要的是,我们想用点和线来
表示一些事物,比如兴趣点、对应点以及检测出的物体。下面是用几个点和一条线
绘制图像的例子

 \begin{figure}[H]
  \centering
    \includegraphics[width=0.9\linewidth]{26.jpg}
  \caption{绘制后的图片}
   \end{figure}
 
\begin{verbatim}
# 绘制图形
    draw = ImageDraw.Draw(image)
    draw.rectangle([50, 50, 2500, 1450], outline="red", width=10)
    image.show()
\end{verbatim}


\subsection{应用滤镜和转化为灰度图像}

\begin{verbatim}
  # 应用滤镜
    # ImageFilter.BLUR是一个预定义的滤镜,用于模糊图像。
    blurred_image = image.filter(ImageFilter.BLUR)
    blurred_image.show()

    # 转换为灰度图像
    gray_image = image.convert('L')
    gray_image.show()
  \end{verbatim}
  
 \begin{figure}[H]
  \centering
    \includegraphics[width=0.9\linewidth]{22.png}
  \caption{模糊后的图片}
   \end{figure}  
   
    \begin{figure}[H]
  \centering
    \includegraphics[width=0.9\linewidth]{23.png}
  \caption{灰度图像}
   \end{figure}  
   
\subsection{图片拼接}

\begin{verbatim}
   # 生成第一张图片:左边是原图的左半部分,右边是灰度图片的右半部分
    try:
        combined_image1 = Image.new('RGB', (width, height))
        left_half_1 = image.crop((0, 0, width // 2, height))
        right_half_1 = gray_image.crop((width // 2, 0, width, height))
        combined_image1.paste(left_half_1, (0, 0))
        combined_image1.paste(right_half_1, (width // 2, 0))
        combined_image1.save('combined_image1.jpg')
        print("第一张图片已生成并保存为 combined_image1.jpg")

        # 生成第二张图片:左边是灰度图片的左半部分,右边是原图的右半部分
        combined_image2 = Image.new('RGB', (width, height))
        left_half_2 = gray_image.crop((0, 0, width // 2, height))
        right_half_2 = image.crop((width // 2, 0, width, height))
        combined_image2.paste(left_half_2, (0, 0))
        combined_image2.paste(right_half_2, (width // 2, 0))
        combined_image2.save('combined_image2.jpg')
        print("第二张图片已生成并保存为 combined_image2.jpg")
  \end{verbatim}
   
 \begin{figure}[H]
  \centering
    \includegraphics[width=0.9\linewidth]{24.jpg}
  \caption{拼接的图片1}
   \end{figure}  
   
    \begin{figure}[H]
  \centering
    \includegraphics[width=0.9\linewidth]{25.jpg}
  \caption{拼接的图片2}
   \end{figure}  


   \subsection{gif图生成}
将之前的example图片与拼接后的两张图片做成gif图
\begin{verbatim}
        # 生成 GIF 动画
        frames = [image, combined_image1, combined_image2]
        frame_durations = [500, 500, 500]  # 每帧显示 500 毫秒
        frames[0].save('animation.gif', save_all=True, append_images=frames[1:], duration=frame_durations, loop=0)
        print("GIF 动画已生成并保存为 animation.gif")
\end{verbatim}

 \section{遇到的问题和学到的经验}

\subsection{问题描述:}
使用 pgrep、kill 命令及其选项(如 SIGTERM 和 SIGKILL)时,需要精确控制哪些进程应该被终止或暂停。
  \subsection{学到的经验}
精确查找:pgrep 命令提供了根据进程名或其他属性查找进程的能力,结合 -l 选项可以显示完整的命令行,有助于确认要操作的进程。
温和终止:默认情况下,kill 命令发送 SIGTERM 信号,请求进程自行终止。这是一种更温和的终止方式,允许进程进行清理操作。
强制终止:如果进程没有响应 SIGTERM,可以使用 kill -9 发送 SIGKILL 信号,强制终止进程。但这种方法不会给进程留下清理的机会。
 \subsection{问题描述:}
命令行环境使用不熟练:
在使用 Ctrl-Z、fg、bg 和 jobs 命令管理进程时,可能会遇到一些困惑,比如如何有效地在后台运行长时间运行的进程,以及如何在需要时恢复它们。
  \subsection{学到的经验}
暂停与恢复:Ctrl-Z 是暂停当前前台进程的快速方法,而 fg 和 bg 命令则允许用户根据需要恢复进程在前台或后台运行。
任务列表:jobs 命令提供了一个清晰的视图,显示了当前终端会话中所有暂停和后台运行的进程。
后台运行:通过在命令末尾添加 \&,可以直接让命令在后台运行,这对于启动长时间运行的进程特别有用。
别名命名:在设置别名时,应避免使用与现有命令或常用工具相同的名称,以防止冲突。
替代方案:考虑使用更独特或更长的别名,如 cdc 或 cd\_to,以避免覆盖。
备份与恢复:在更改重要配置文件(如 .bashrc 或 .bash\_profile)之前,最好备份原始文件,以便在需要时恢复。
 \subsection{问题描述:}在尝试使用命令行安装Python库(如使用pip)时,经常遇到权限问题或路径错误,导致安装失败。
 \subsection{学到的经验}
  学会了使用sudo(在Linux/macOS上)或管理员命令提示符(在Windows上)来提升权限,以及使用虚拟环境(如venv或conda)来隔离和管理不同项目的依赖。

 \subsection{问题描述:}Python基础语法混淆:在编写Python脚本时,经常混淆变量作用域、列表和元组的区别,以及循环和条件语句的嵌套。
学到的经验:通过大量练习和阅读官方文档,加深了对Python基础语法的理解,学会了使用调试工具(如pdb)来追踪和解决问题。

\subsection{问题描述:}
视觉分析库的使用难题:
在使用Matplotlib、Pandas或OpenCV等库进行图像处理和可视化时,遇到性能瓶颈或难以实现的特定效果。
  \subsection{学到的经验}
  学习了如何优化图像加载和处理过程,如使用适当的数据类型、减少不必要的计算以及利用并行处理。同时,也学会了查看官方文档和社区论坛,寻找解决特定问题的技巧和最佳实践。

学到的经验总结
持续学习:技术日新月异,保持对新技术和新方法的关注和学习是非常重要的。
实践出真知:理论知识是基础,但只有通过实践才能真正掌握和应用。
社区资源:充分利用官方文档、社区论坛和开源项目等资源,可以更快地解决问题并提升技能。
耐心和毅力:面对问题和挑战时,保持耐心和毅力是成功的关键。

 \section{代码链接}
      请访问以下链接来查看我在Github上的相关练习、报告和代码:
\url{https://github.com/boken319/git_push}
我的在线Latex文件的网址:
\url{https://cn.overleaf.com/read/ystjjrypyfnf#402b9a}

\end{document}
